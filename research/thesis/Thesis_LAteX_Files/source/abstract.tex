% In total max. 1 Page!
\AMstudentthesisAbstract{%
%
% Abstract English:
The development of exoskeletons for the rehabilitation and assistance of disabled people has necessitated the development of high-performance controllers to reproduce complex movements. The addition of Machine Learning has opened up the possibilities for higher-performance controllers. In the case of a lower limb exoskeleton, one of the primary objectives of the high-level controller is to ensure consistent gait. This means generating gait trajectories, or controlling gait at every instant. In this work, carried out in collaboration with the TUM DASH student initiative, the aim is to create a high-level controller which, using Machine Learning methods such as Deep Neural Networks and Gaussian Process Regression, will generate gait trajectories for the hip and knee. Trajectories here are the set of positions the joint must take to complete a gait. These trajectories are generated by generating isu key points of the trajectory before being interpolated. This trajectory generation makes it possible to obtain an average relative RMSE of less than 10\% compared to the real amplitude of the trajectory.
}{%
%
% Zusammenfassung Deutsch:
Die Entwicklung von Exoskeletten für die Rehabilitation und Unterstützung von Menschen mit Behinderungen erforderte die Entwicklung von leistungsfähigeren Controllern, um komplexe Bewegungen nachvollziehen zu können. Das Hinzufügen von Machine Learning hat die Möglichkeiten für leistungsfähigere Controller auf hohem Niveau eröffnet. Im Falle eines Lower Limb Exoskeleton besteht eines der Hauptziele des High-Level-Controllers darin, einen konsistenten Gang zu gewährleisten. Dies geschieht durch die Erzeugung von Gangtrajektorien oder die Kontrolle des Gaits zu jedem Zeitpunkt. In dieser Arbeit, die in Zusammenarbeit mit der Studenteninitiative TUM DASH entstanden ist, soll ein High-Level-Controller entwickelt werden, der mithilfe von Machine-Learning-Methoden wie Deep Neural Networks und Gaussian Process Regression Gangtrajektorien für Hüfte und Knie generiert. Trajektorien sind hier die Gesamtheit der Positionen, die das Gelenk einnehmen muss, um einen Gang zu vollenden. Die Generierung dieser Trajektorien erfolgt über die Generierung von isu-Schlüsselpunkten der Trajektorien, bevor sie interpoliert werden. Diese Trajektoriengenerierung ermöglicht es, einen durchschnittlichen relativen RMSE von weniger als 10\% im Vergleich zur tatsächlichen Amplitude der Trajektorie zu erhalten.
}%
%
%
